

\subsection{Distribuição Poisson}

%%%-------------------------------------------------------------------
\begin{frame}{Distribuição Poisson}
\begin{itemize}
\item Função de probabilidade
\begin{eqnarray}
p(y;\mu) &=& \frac{\mu^y}{y!}\exp\{-\mu\} \nonumber \\
	     &=& \frac{1}{y!} \exp \{\phi y -  \exp\{\phi\} \}, \quad y \in \mathbb{N}_{0},
\end{eqnarray}
onde $\phi = \log \{\mu\} \in \mathbb{R}$ e $\kappa(\phi) = \exp\{\phi\}$
denota a função cumulante.
\vspace{0,5cm}
\item $\mathrm{E}(Y) = \kappa^{\prime}(\phi) = \exp\{\phi\} = \mu$. 
\vspace{0,5cm}
\item $\mathrm{var}(Y) = \kappa^{\prime \prime}(\phi) = \exp\{\phi\} = \mu$.
\vspace{0,5cm}
\item Em \texttt{R} temos \texttt{dpois()}.

\end{itemize}
\end{frame}

%%%-------------------------------------------------------------------
\begin{frame}{Regressão Poisson}
\begin{itemize}
\item Considere $(y_i, x_i)$, $i = 1,\ldots, n$, onde $y_i$'s são iid 
realizações de $Y_i$ de acordo com a distribuição Poisson.
\vspace{0,5cm}
\item Modelo de regressão Poisson
$$Y_i \sim P(\mu_i), \quad  \text{sendo} \quad \mu_i = g^{-1}(\boldsymbol{x_i}^{\top} \boldsymbol{\beta}),$$
onde $\boldsymbol{x_i}$ and $\boldsymbol{\beta}$ são vetores $(p \times 1)$
de covariáveis conhecidas e parâmetros de regressão, respectivamente.
\vspace{0,5cm}
\item Em \texttt{R} temos \texttt{glm(..., family = poisson)}.
\end{itemize}
\end{frame}
